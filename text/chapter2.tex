\section{Grenzen relationaler Datenbanken}

% Moderne Big-Data Anwendungen fordern skalierbare und flexible Datenspeicher um mit der wachsenden Menge dynamischer Daten mitzuhalten. Trotz kontinuierlicher Weiterentwicklung und Optimierung stoßen relationale Datenbanken inbesondere hinsichtlich Skalierbarkeit und Flexibilität an ihre Grenzen. Dieses Kapitel beschreibt die Hintergründe dieser Limitierungen und erklärt die Notwendigkeit alternativer Datenbanktechnologien.

Moderne Big-Data-Anwendungen erfordern skalierbare und flexible Datenspeicher, um mit der wachsenden Menge dynamischer Daten Schritt zu halten. Trotz kontinuierlicher Optimierung stoßen relationale Datenbanken dabei zunehmend an ihre Grenzen. Dieses Kapitel erläutert die Hintergründe dieser Limitierungen und unterstreicht dabei die Notwendigkeit und Einsatzbereiche alternativer Datenbanktechnologien.

\subsection{Fehlende Flexibilität \& Impedance Mismatch}

Bei der Erstellung von Tabellen in einer relationalen Datenbank müssen Spalten und Datentypen in einem Schema definiert werden. Zur Vermeidung von Redundanzen und zur Sicherstellung der Datenintegrität werden komplexe Datenmodelle zunächst normalisiert. Dabei wird das Modell auf mehrere Tabellen verteilt und mit Primär- und Fremdschlüsseln verknüpft, wodurch auch komplexe Objektstrukturen auf einfache Tabellen abbildbar sind. Die strikte Schematisierung stellt sicher, dass nur Datensätze akzeptiert werden, die der definierten Struktur und Vorgaben des Schemas entsprechen. Weicht die Länge oder der Datentyp eines Wertes ab, wird der Datensatz nicht akzeptiert. Die resultierende Datenkonsistenz ist eine der zentralen ACID-Eigenschaften relationaler Datenbanken und maßgeblicht verantwortlich für die Verlässlichkiet und weite Verbreitung dieser Systeme. \footcite[S. 3]{phiriComparativeStudyNoSQL2017}

Die Anforderungen moderner Software-Anwendungen haben sich im Laufe der Zeit stark gewandelt. Nutzer cloudbasierter Software-as-a-Service-Lösungen fordern kontinuierliche Wartung und Erweiterung ihrer Software. Die heute verbreitete agile Softwareentwicklung setzt hierfür auf einen iterativ-dynamischen Entwicklungsprozess, der häufig  Änderungen oder Erweiterungen des Datenmodells erfordert. Zwar unterstützen moderne relationale Datenbanksysteme die Modifikation bestehender Tabellen, doch diese erfordert besondere Vorsicht und bietet nicht der gewünschten Flexibilität und Stabilität. \footcite[S. 197]{harrisonNextGenerationDatabases2015} Darüber hinaus sind Big-Data-Anwendungen zunehmend mit den Herausforderung der effizienten Verwaltung riesiger Mengen unstrukturierter oder semi-strukturierter Daten konfrontiert. Große Anbieter haben die strikte Schematisierung des relationalen Modells als einschränkenden Faktor erkannt und suchen nach Alternativen, welche die starren Strukturen aufbrechen vereinfachen.

Ein weiteres Problem in diesem Kontext, das Entwickler schon seit Jahrzehnten beschäftigt, ist der Impedance Mismatch. Der Begriff beschreibt den grundlegenden Unterschied zwischen dem objektorientierten Datenmodell und dem relationalen Modell. Während Anwendungen häufig mit komplexen Objektstrukturen arbeiten, die Vererbung und Typisierung unterstützen, basieren RDBMS auf einfachen Tabellenstrukturen. Dieser Modellkontrast führt zu einem hohen Übersetzungsaufwand zwischen Anwendung und Datenbank. \footcite{newardVietnamComputerScience2006} Die Persistierung komplexer Objekte in relationaler Form lässt sich mit einem Parkhaus vergleichen, in dem Autos in ihre Einzelteile zerlegt gelagert werden. Jeder Lese- oder Schreibzugriff erfordert das Zusammensetzen beziehungsweise Zerlegen der Objekte. Reine Objektdatenbanken, die dieses Problem lösen sollten, konnten sich nicht durchsetzen, sodass moderne RDBMS heute objektorientierte Konzepte unterstützen. Auch der Einsatz sogenannter objektrelationaler Abbildungen (ORM) kann helfen, die Auswirkungen des Impedance Mismatch zu minimieren, indem er den Übersetzungsschritt automatisieren. Das grundlegende Problem jedoch bleibt: Objektstrukturen sowie semi-strukturierter Daten sind eine grundlegende Schwäche relationaler Datenbanken.

\subsection{Eingeschränkte Skalierbarkeit}

Das rasante Datenwachstum der letzten zwei Dekaden bedingte die Notwendigkeit kontinuierlich expandierender digitaler Infrastruktur. Die parallel zunehmende Rechenleistung und Speicherkapazitäten einzelner Computerchips und Festplatten konnten diesem Trend nicht kompensieren, sodass die vertikale Skalierung durch Erhöhung der Rechenleistung einzelner Computer an ihre technischen und wirtschaftlichen Grenzen stieß. Um das Problem zu lösen, wurde vermehrt auf die horizontale Skalierung gesetzt, bei der mehrere physikalisch getrennte Rechenknoten zusammenarbeiten. Die genannten Limitierungen konnten durch Parallelisierung gelöst werden und ermöglichen theoretisch unbegrenzte Kapazität. Darüber hinaus führt die Verteilung zu einer höheren Verfügbarkeit und Ausfallsicherheit. Ist ein Knoten, beispielsweise auf Grund von Wartungsarbeiten nicht erreichbar, wird die verlorene Rechenleistung von anderen Knoten ausgeglichen, ohne merkliche Auswirkungen auf die Anwendung und deren Nutzer. 

Relationale Datenbanken sind traditionell für die vertikale Skalierung ausgelegt, und speziell optimiert für den Betrieb auf einem einzelnen Server. % \footcite{}
Mit den wachsenden Anforderungen wurden jedoch Ansätze wie entwickelt, um auch die horizontale Skalierbarkeit zu ermöglichen: Der sogenannte Memcache verlagerte die typischerweise häufiger auftretenden Lesezugriffe auf mehrere Replikationsserver. Zusätzlich wurden beim Sharding gesamte Datenbanken partitioniert und auf mehrere Knoten verteilt. \footcite[S. 16]{dertingerNoSQLDatenbankenUndNichtrelationale2025}. Das CAP-Theorem besagt, kein verteiltes System könne gleichzeitig \textbf{C}onsistency (Konsistenz), \textbf{A}vailability (Verfügbarkeit) sowie \textbf{P}artition Tolerance (Ausfalltoleranz) erfüllen. Die Ausfalltoleranz ist bei in verteilten Systemen stets besonders wichtig, da sonst der Ausfall eines Knoten das gesamte System beeinträchtigt. Demnach muss in diesem Fall zwischen Verfügbarkeit und Konsistenz gewählt werden - Weil die Verfügbarkeit meist Priotität hat, kann eine grundlegende ACID-Eigenschaft bei der Verteilung relationaler Datenbanken nicht mehr garantiert werden.




Die effektive und effiziente (optimale) horizontale Skalierung relationaler Datenbanken ist aufgrund der strengen ACID-Eigenschaften und der Einschränkung des CAP-Theorems nicht, oder nur mit erheblichem Kostenaufwand, möglich. \footcite[S. 1]{schreinerWhenRelationalBasedApplications2019} 

Die ACID-Eigenschaften werden bei großen Datenmengen zu einem zunehmend limitierenden Hindernis, welches.

Es stellte sich jedoch heraus, dass diese komplizierten Ansätze nicht nur viele Nachteile und hohe Kosten verursachen, sondern die notwendige Skalierbarkeit, wie etwa von Sozialen Netzwerken oder Online-Shops benötigt, nicht erreichbar ist. \footcite[S. 41-43]{harrisonNextGenerationDatabases2015}. 







% \newpage


\section{Einleitung}

Die Raumfahrt ist eines der komplexesten und kostspieligsten Projekte in der Menschheitsgeschichte, das bedeutende technische Innovationen erfordert. Historisch gesehen fand eine systematische Technikfolgenabschätzung (TA) in der Raumfahrt lange Zeit nicht statt. Die beiden führenden Raumfahrtnationen – die USA und die Sowjetunion – instrumentalisierten die Raumfahrt im Kontext ihres geopolitischen Wettlaufs. In dieser Phase folgte die technologische Entwicklung primär einer politischen Logik, wobei politische Entscheidungen den Kurs der technologischen Entwicklung bestimmten, während ökologische, ökonomische und gesellschaftliche Auswirkungen vernachlässigt wurden. Die Ära des Raumfahrtaufruchs begann 1957 mit dem sowjetischen Satelliten Sputnik 1, gefolgt von weiteren Meilensteinen wie dem ersten Menschen im All, Juri Gagarin, und der Mondlandung durch Neil Armstrong 1969. \footcite{adamsAnhalterDurchGalaxis2013}

% \subsection{Der Begriff \glqq New Space\grqq}

\subsection{Motivation \& Aufbau der Arbeit}

\subsection{Big Data und relationale Datenbanken}








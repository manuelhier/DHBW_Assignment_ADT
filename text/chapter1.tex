\section{Einleitung}

Die über 50 Jahre alte Technologie durchlief im Laufe der Jahrzehnte mehrere Evolutionszyklen, um mit den Wandelnden Anforderungen mitzuhalten. Zusammen mit mit der Structured Query Language (kurz: \textbf{SQL}) bilden relationale Datenbanken bis heute den de facto  Standard für das Speichern und Verwalten digitaler Daten und Informationen aller Art und Güte. In der montalich aktualisierten Rangliste der Popularität verschiedener DBMS der Webseite \textit{db-engines.com} sind im März 2025 sieben der 10 meistgenutzten Datenbanken von primär relationaler Natur. \footcite{redgatesoftwareltd.DBEnginesRanking2025}

% ChatGPT
Trotz kontinuierlicher Weiterentwicklungen und Optimierungen hinsichtlich Skalierbarkeit, Performance und Flexibilität stoßen relationale Datenbanken bei bestimmten modernen Anwendungsfällen an ihre Grenzen. In diesem Kontext haben sich alternative Datenbanktechnologien wie NoSQL- und NewSQL-Systeme entwickelt, die speziell für verteilte, flexible oder hochgradig skalierbare Datenverarbeitung konzipiert wurden.

% \subsection{Motivation \& Aufbau der Arbeit}

\subsection{Relationale Datenbanken und SQL}

Das relationale Modell in Verbindung mit der Structured Query Language (SQL)  bildet die Grundlage relationaler Datenbankmanagementsysteme (RDBMS) und wurde im Jahr 1970 erstmals durch den Mathematiker Edgar F. Codd konzeptioniert und vorgeschlagen. \footcite{coddRelationalModelData1970} Seine Forschungsarbeit basiert auf der Mengenlehre, einem Teilgebiet der Mathematik. Eine relationale Datenbank ist demnach eine Menge von Tabellen, den sogenannten Relationen. Das Relationenschema definiert die Anzahl und Art der möglichen Spalten bzw. Attribute einer Tabelle und wird bei Erzeugung der Tabelle festgelegt. Eine Zeile der Tabelle, bildet einen Datensatz. Diese werden zeilenweise in Form von Tupeln in die Tabellen eingefügt und müssen dabei den Regeln des Relationenschemas folgen. Ein weiterer Teil Codd's Forschung war die Entwicklung der relationalen Algebra, deren Operationen die Grundlage der universellen Abfragesprache SQL bilden. Diese kombiniert mengenorientierte Operatoren wie die Vereinigungsmenge oder das Kartesische-Produkt mit den relationenorientierten Operatoren Selektion und Projektion. Die Datenabfrage mit SQL folgt dem SELECT-FROM-WHERE Ausdruck, mit welchem Daten von einer oder mehreren Tabellen unter Selektions- und Filterbedingungen abgefragt und tabellarisch angezeigt werden. Auf Basis dieser Forschungsarbeit entstanden in den folgenden Dekaden zahllose Datenbanksysteme wie beispielsweise IBM's Db2, die Oracle Database und der Microsoft SQL Server. \footcite[S. 15-24]{meierWerkzeugeDigitalenWirtschaft2018}

%Des Weiteren bietet die große Auswahl an Implementierungen verschiedener Hersteller ein breites Spektrum an Auswahl und Expertenwissen. Für die Mehrheit der 

\subsection{Big Data - Große Daten, große Probleme?}

Der Begriff Big Data (dt. Große Daten) beschreibt den allgegenwärtigen Trend wachsender Datenmengen. Hierbei liegt der Fokus nicht lediglich auf der Wachsenden Größe einzelner Datensätze wie die Übersetzung vermuten ließe, sondern vielmehr der horizontalen Skalierung und somit einer überwältigenden Anzahl von Datensätzen welche in Echtzeit verarbeitet und gespeichert werden müssen. Passende Beispiele hierfür sind beispielsweise der Datenfluss großer Online-Shops, Soziale-Netzwerke und digitale Streaming-Platformen. \footcite[S. 5]{meierWerkzeugeDigitalenWirtschaft2018}
\section{Einleitung}

Relationale Datenbanken und die Abfragesprache \ac{sql} bilden seit Jahrzehnten den de facto Standard für das Speichern, Verwalten und Abfragen digitaler Daten und Informationen aller Art und Güte. In der montalich aktualisierten Rangliste der Popularität verschiedener \ac{dbms} der Webseite \textit{db-engines.com} sind im März 2025 sieben der 10 meistgenutzten Datenbanken von primär relationaler Natur. \footcite{redgatesoftwareltd.DBEnginesRanking2025} Um mit den wandelnden Anforderungen mitzuhalten durchlief die über 50 Jahre alte Technologie mehrere Evolutionszyklen. Trotz Optimierungsversuchen hinsichtlich Skalierbarkeit, Performance und Flexibilität stoßen relationale Datenbanken 
% im Einsatz in den größten Webseiten und Datenplattformen
in diesen Bereichen zunehmend an ihre Grenzen. In diesem Kontext wurden alternative Datenbanktechnologien entwickelt, die speziell für verteilte, flexible und hochgradig skalierbare Datenverarbeitung ausgelegt sind. \footcite[S. 13-17]{harrisonNextGenerationDatabases2015}

% Die über 50 Jahre alte Technologie durchlief im Laufe der Jahrzehnte mehrere Evolutionszyklen, um mit den Wandelnden Anforderungen mitzuhalten. Zusammen mit mit der Structured Query Language (kurz: \textbf{SQL}) bilden relationale Datenbanken bis heute 

% ChatGPT
% Trotz kontinuierlicher Weiterentwicklungen und Optimierungen hinsichtlich Skalierbarkeit, Performance und Flexibilität stoßen relationale Datenbanken bei bestimmten modernen Anwendungsfällen an ihre Grenzen. In diesem Kontext wurden alternative Datenbanktechnologien wie NoSQL- und NewSQL-Systeme entwickelt, die speziell für verteilte, flexible oder hochgradig skalierbare Datenverarbeitung ausgelegt sind.

% \subsection{Motivation \& Aufbau der Arbeit}

\subsection{Relationale Datenbanken und SQL}

Das relationale Modell in Verbindung mit SQL bildet die Grundlage \ac{rdbms} und wurde im Jahr 1970 erstmals durch den Mathematiker Edgar F. Codd konzeptioniert und vorgeschlagen. \footcite{coddRelationalModelData1970} Seine Forschungsarbeit basiert auf der Mengenlehre, einem Teilgebiet der Mathematik. \footcite[S. 66]{langerDevelopingPathData2023} Eine relationale Datenbank ist demnach eine Menge von Tabellen, den sogenannten Relationen. Das Relationenschema definiert die Anzahl und Art der möglichen Spalten bzw. Attribute einer Tabelle und wird bei Erzeugung der Tabelle festgelegt. Eine Zeile der Tabelle, bildet einen Datensatz. Diese werden zeilenweise in Form von Tupeln in die Tabellen eingefügt und müssen dabei den Regeln des Relationenschemas folgen. Ein weiterer Teil Codd's Forschung war die Entwicklung der relationalen Algebra, deren Operationen die Grundlage der universellen Abfragesprache SQL bilden. Diese kombiniert mengenorientierte Operatoren wie die Vereinigungsmenge oder das Kartesische-Produkt mit den relationenorientierten Operatoren Selektion und Projektion. Die Datenabfrage mit SQL folgt dem SELECT-FROM-WHERE Ausdruck, mit welchem Daten von einer oder mehreren Tabellen unter Selektions- und Filterbedingungen abgefragt und tabellarisch angezeigt werden. Auf Basis dieser Forschungsarbeit entstanden in den folgenden Dekaden zahllose Datenbanksysteme wie beispielsweise IBM's Db2, die Oracle Database und der Microsoft SQL Server. \footcite[S. 15-24]{meierWerkzeugeDigitalenWirtschaft2018}

%Des Weiteren bietet die große Auswahl an Implementierungen verschiedener Hersteller ein breites Spektrum an Auswahl und Expertenwissen. Für die Mehrheit der 

\subsection{Big Data - Große Daten, große Probleme?}

Der Begriff \textbf{Big Data} (dt. \enquote{Große Daten}) beschreibt den gegenwärtigen Trend stetig wachsender Datenmengen. Eine einheitliche Definition existiert in der Literatur nicht. Stattdessen werden häufig lediglich charakteristischen Merkmale anhand der drei V's erläutert: Eine enorme Datenmenge (\emph{Volume}) aus unterschiedlichen Quellen und Formaten (\emph{Variety}) muss in hoher Geschwindigkeit (\emph{Velocity}) verarbeitet und gespeichert werden. \footcite[S. 5]{meierWerkzeugeDigitalenWirtschaft2018} Typische Anwendungsfälle sind große E-Commerce-Plattformen, soziale Netzwerke und digitale Streaming-Dienste. Neuere Definitionen von Big Data beziehen zudem den Unternehmenswert (\emph{Value}) mit ein, der den wirtschaftlichen Nutzen und die Relevanz solcher Technologien unterstreicht. Die Extraktion dieses Mehrwerts aus der Datenflut ist eine zentrale Herausforderung und wird treffend unter dem Motto \emph{„Extracting Value from Chaos“} zusammengefasst. \footcite[S. 173 ff.]{chenBigDataSurvey2014} Bereits im Jahr 2011 definierten \emph{Manyika et al.} in einem Bericht des McKinsey Global Institute Big Data als Datenmengen, die die Verarbeitungsgrenzen herkömmlicher Datenbanksysteme überschreiten. Sie verzichteten dabei bewusst auf die Definition einer quantitative Grenze – etwa in Terabyte oder Petabyte – in der Annahme, dass sich diese Schwelle mit dem technologischen Fortschritt stetig verschiebt. \footcite{manyikaBigDataNext2011} Die größten Technologieunternehmen wie Google und Facebook, aber auch der Versandgigant Amazon wurden durch wachsende Datenmengen schnell mit den Grenzen und Limitierungen traditioneller Datenbanken konfrontiert. Um Big Data effektiv und gewinnbringend einzusetzen arbeiteten sie  an innovativen Lösungen und Ansätzen, was die Datenspeicherung nachhaltig veränderte und eine neue Generation von Datenbanken einleitete.


% Hierbei liegt der Fokus nicht lediglich auf der Wachsenden Größe einzelner Datensätze wie die Übersetzung vermuten ließe, sondern ebenso auf der horizontalen Skalierung und somit einer überwältigenden Anzahl von Datensätzen welche in Echtzeit verarbeitet und gespeichert werden müssen. 

% "Over the past few years, nearly all major companies, including EMC, Oracle, IBM, Microsoft, Google, Amazon, and Facebook, etc. have started their big data projects. Taking IBM as an example, since 2005, IBM has invested USD 16 billion on 30 acquisitions related to big data." Chen S, 175

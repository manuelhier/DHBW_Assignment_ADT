\section{Zusammenfassung und Ausblick}

Diese Arbeit hat die Grenzen relationaler Datenbanken im Kontext wachsender Datenmengen in modernen Big-Data-Anwendungen untersucht. Während relationale Datenbanken lange Zeit das Standardwerkzeug für die Datenspeicherung waren, führten Skalierbarkeits- und Flexibilitätsprobleme zur Entwicklung alternativer Systeme. NoSQL-Datenbanken wurden eingeführt, um diese Herausforderungen zu bewältigen, brachten jedoch neue Probleme, wie fehlende Standardisierung und Konsistenz, mit sich. Als ein hybrider Ansatz entstand NewSQL, welcher die Vorteile relationaler Datenbanken mit der Skalierbarkeit von NoSQL kombiniert. Heute zeigt sich eine zunehmende Konvergenz zwischen RDBMS, NoSQL und NewSQL. Die zunehmende Adaption von ACID-Transaktionen in NoSQL-Systemen sowie die Erweiterung relationaler Datenbanken um Schema-Flexibilität verdeutlichen, dass sich die Grenzen zwischen den Technologien immer weiter auflösen. Dieser Wandel zeigt, dass keine einzelne Lösung für alle Anwendungsfälle optimal ist – vielmehr ist die Wahl des passenden Datenbankmodells stark von den spezifischen Anforderungen einer Anwendung abhängig.

Die Zukunft der Datenbanken wird von hybriden Architekturen geprägt, die verschiedene Modelle innerhalb eines Systems vereinen. Insbesondere die rasante Entwicklung von Big Data und künstlicher Intelligenz könnte die Datenbanktechnologien in den nächsten Jahren weiter vorantreiben. Langfristig könnten universelle Datenbanken entstehen, die sich flexibel an unterschiedliche Anwendungsfälle anpassen und sowohl relationale als auch nicht-relationale Strukturen effizient verwalten. Die bereits existierenden Multi-Modell-Datenbanken bilden hierfür ein solides Fundament, welches es zu perfektionieren gilt.
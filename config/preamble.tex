%%%%%%%%%%%%%%%%%%%%%%%%%%%%%%%%%%%%%%%%%%%%%%%%%%%%%%%%%%%%%%%%%%%%%%%%%%%%%%%
%% 2020-06-20
%% Descr:       Formale Einstellungen für die Praxisarbeit
%% Author:      Vorlage erstellt von Daniel Spitzer an der DHBW Lörrach 
%% Angepasst:   Katja Wengler, ZWI, DHBW Karlsruhe
%%%%%%%%%%%%%%%%%%%%%%%%%%%%%%%%%%%%%%%%%%%%%%%%%%%%%%%%%%%%%%%%%%%%%%%%%%%%%%%

\documentclass[a4paper,12pt]{article}

% Sprache & Darstellung
\usepackage[english, ngerman]{babel} % Hier ggf. Sprache anpassen
\usepackage{lmodern}

% Babel Ersatz
% \usepackage{polyglossia}
% \setdefaultlanguage[spelling=new, babelshorthands=true]{german}

% Schriftart Carlito, fast identisch mit Calibri. Calibri ist auf Linux und MacOS nicht verfügbar. 
% \usepackage{carlito}
% \setmainfont{carlito}

% Andere Schriftarten
% \usepackage{fontspec}
% \setsansfont{Arial}
% \setmonofont{Courier New}

% Seitenränder Projekt bzw. Bachlorarbeit
% \usepackage[left=2.5cm, right=2.5cm, head=1.25cm, bottom=2cm, foot=1.25cm, includefoot]{geometry}

% Seitenränder Links / Rechts Symmetrisch z.B. für Portfolio oder Assignments
\usepackage[left=2.5cm, right=2.5cm, head=1.25cm, bottom=2cm, foot=1.25cm, includefoot]{geometry}

% Zeilenabstand: 1.5
\usepackage{setspace}
\setstretch{1.5}

% Inhaltsverzeichnis
\usepackage[nottoc]{tocbibind}
\usepackage{parskip}

% Abkürzungsverzeichnis
\usepackage[nohyperlinks, withpage, smaller, printonlyused]{acronym}

% Literaturverzeichnis
\usepackage[backend=biber, style=apa, language=german, block=space, date=year]{biblatex}
\addbibresource{literature/bibliography.bib}

% Fuß und Kopfzeilen
\usepackage{fancyhdr}
\renewcommand{\headrulewidth}{0pt}
\renewcommand{\footrulewidth}{0pt}
\def \footer{
\begin{center}
\hfill
%\fontsize{\footerFontSize}{\footerFontSize}\selectfont\thesisFooterTitle
\hfill
\thepage
\end{center}
}

% Positioniert die Fußnoten fest am unteren Ende der Seite
\usepackage[hang, flushmargin, bottom, multiple]{footmisc}
\renewcommand{\hangfootparindent}{0.5em}
\renewcommand{\footnotelayout}{\hspace{0.5em}}

% Tabellen & Abbildungen
\usepackage{tabulary}
\usepackage{tabularx}
\usepackage{float}  % Positioniert Tabellen und Abbildungen
\usepackage{graphicx}
\graphicspath{ {images/} }
\usepackage[labelfont=bf]{caption}
\captionsetup{belowskip=0pt}
%\captionsetup{belowskip=-9pt}

% Aufzählungen
\usepackage{enumitem}

% Code-Blöcke
\usepackage{minted}

% Mathematik 
\usepackage{mathtools}
\usepackage{amssymb}

% Anführungszeichen mit \enquote{...}
\usepackage{csquotes}

% Das Paket hyperref muss als letztes Paket geladen werden. Das Paket setzt Links im PDF-Dokument.
\usepackage[hidelinks, unicode]{hyperref}
\hypersetup{pdftitle = {\thesisTitle}, pdfauthor = {\name}}

% Normale Schriftart für URLs
\renewcommand{\UrlFont}{}

% Variable für das Speichern der Seitenzahl (römisch -> arabisch -> römisch)
\newcounter{pageNumber}

% Nummerierung: 2.1, 2.2 usw.
\renewcommand{\labelenumii}{\theenumii}
\renewcommand{\theenumii}{\theenumi.\arabic{enumii}.}

% Befehl für ein einleitendes Zitat
\usepackage{csquotes}
\newcommand{\epigraph}[2]{
    \begin{quote}\begin{quote}
        \begin{center}
            \textit{#1}
        \end{center}
        \hfill #2
    \end{quote}\end{quote}
}